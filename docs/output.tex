\documentclass{article}
\usepackage{amsmath}

\begin{document}

\title{Brief explanation}
\maketitle


\section{Current-current correlation function}
Current operator $\hat{J}$: see below of Eq.3 of https://arxiv.org/abs/1807.01625.
Current-current correlation function $C_{J}$ is defined as below
\begin{equation}
C_{J} = \langle \Psi | \hat{J}\hat{J} | \Psi \rangle
\end{equation}
(where $|\Psi\rangle$ is a wavefunction of system)
In this program, $C_{J}$ is limited to a case with same time and same position (i.e. $JJ$ has only two AO indices)

\section{Spin-spin correlation function}
The z component of the spin operator is given below, ignoring the constant multiple terms.
\begin{equation}
    S^{z} = \sum_{i} (a^{\dagger}_{i\alpha}a_{i\alpha} - a^{\dagger}_{i\beta}a_{i\beta})
\end{equation}

Spin-spin correlation function is given below:
\begin{equation}
C_{s} = \langle \Psi | \hat{S}^{z}\hat{S}^{z} | \Psi \rangle
\end{equation}

\section{Charge-charge correlation function}
Occupation number operator at site $i$ is given below:
\begin{equation}
\hat{n}_{i} = \hat{a}^{\dagger}_{i\alpha}\hat{a}_{i\alpha} + \hat{a}^{\dagger}_{i\beta}\hat{a}_{i\beta}
\end{equation}
Charge-charge correlation function is given below:
\begin{equation}
C_{n}^{ij} = \langle \Psi | \hat{n}_{i}\hat{n}_{j} | \Psi \rangle
- \langle \Psi | \hat{n}_{i} | \Psi \rangle
\langle \Psi | \hat{n}_{j} | \Psi \rangle
\end{equation}


\section{Exciton correlation}
The equation for random phase approximation (RPA) is below:
\begin{equation}
\begin{pmatrix}
A & B \\
-B^* & -A^*
\end{pmatrix}
\begin{pmatrix}
X \\
Y
\end{pmatrix}
= \omega
\begin{pmatrix}
1 & 0 \\
0 & -1
\end{pmatrix}
\begin{pmatrix}
X \\
Y
\end{pmatrix}
\end{equation}

The RPA wavefunction is below:
\begin{equation}
|\Phi \rangle = \sum_{mi} (X_{mi}|mi\rangle - Y_{mi}|im\rangle)
\end{equation}

Note that $m$ is the index of the occupied moleculart orbital, $i$ is the index of the virtual molecular orbital,
and $|mi\rangle$は$\hat{a}^{\dagger}_{i}\hat{a}_{m}|\mathrm{HF}\rangle$
Convert $X_{mi}$ to AO basis with molecular orbital coefficients to obtain exciton correlation.

\section{Green's function for a mean-field calculation}
 See Eq.7 of https://arxiv.org/abs/2002.05875.
 Note in this program, real space not reciprocal space is treated.

\end{document}
